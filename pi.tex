\documentclass[a4paper,12pt]{article}
\usepackage[utf8]{inputenc}
\usepackage[spanish]{babel}
\usepackage{graphicx}

\begin{document}

\title{Número PI}
\author{Aidé Alicia Cordobés Betancor \\
       Técnicas Experimentales~\footnote{Artículo sobre el número PI}
       }
\date{09/04/2014}
\maketitle

\begin{abstract}
$pi$ (pi) es la relación entre la longitud
de una circunferencia y su diámetro,
en geometría euclidiana. Es un número
irracional y una de las constantes matemáticas
más importantes. Se emplea frecuentemente en
matemáticas, física e ingeniería.
El valor numérico de $pi$, truncado a sus primeras
cifras, es el siguiente:
    
    pi = 3,14159265358979323846 

    \end{abstract}
\begin{figure}[t]
\includegraphics[scale=0.1]{imagen1.eps}
\caption{Ejemplo de gráfica}
\label{fig1}
\end{figure}
\section{Un poco de historia}
La búsqueda del mayor número de decimales
del número $pi$ ha supuesto un esfuerzo constante
de numerosos científicos a lo largo de la historia.
Algunas aproximaciones históricas de pi son las
siguientes.
\subsection{Egipto}
El valor aproximado de pi en las antiguas
culturas se remonta a la época del escriba
egipcio Ahmes en el año 1800 a. C.,
descrito en el papiro Rhind,3 donde se emplea
un valor aproximado de pi afirmando que el área
de un círculo es similar a la de un cuadrado
cuyo lado es igual al diámetro del círculo
disminuido en 1/9; es decir, igual a 8/9 del
diámetro.
\subsection{Matemática China}
El cálculo de pi fue una atracción para los
matemáticos expertos de todas las culturas.
Hacia 120, el astrónomo chino Zhang Heng (78-139)
fue uno de los primeros en usar la aproximación 
raíz de 10, que dedujo de la razón entre el volumen
de un cubo y la respectiva esfera inscrita.
Un siglo después, el astrónomo Wang Fang lo
estimó en 142/45 (3,155555), aunque se desconoce
el método empleado.
\section{Características matemáticas}
\subsection{Definiciones}
$pi$ es la relación entre la longitud de una
circunferencia y su diámetro.
Por tanto, también $pi$ es:
El área de un círculo unitario
(de radio unidad del plano euclídeo).
El menor número real x positivo tal que
sin(0)=0


En la figura \ref{fig1} se puede ver una
imagen del número PI.

En la figura \ref{table1} se muestra el
número de chicos y chicas que hay en dos
clases distintas.
\begin{table}[h]
\begin{center}
\begin{tabular}{|l|c|c|c|}
\hline
Clase & Chicos & Chicas & Total \\ \hline
Clase 1 & 30 & 25 & 55 \\ \hline
Clase 2 & 15 & 20 & 35 \\ \hline 
\end{tabular}
\caption{Mi tabla}
\label{table1}
\end{center}
\end{table}

\begin{thebibliography}{00}
\bibitem{Lam:86}
http://es.wikipedia.org/wiki/N%C3%BAmero_%CF%80
\bibitem{Lam:86}
http://centros5.pntic.mec.es/ies.de.bullas/dp/matema/conocer/numpi.htm
\end{thebibliography}

\end{document}

